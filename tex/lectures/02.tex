\chapter{Покращення якості зображення}

\shortLectureDescription{Розглянемо методи, за допомогою яких можна надати зображенню такої якості, щоби воно сприймалося людиною більш комфортно. Часто корисним буває підкреслити деякі риси, нюанси картинки, щоб покращити її суб'єктивне сприйняття. }

\section{Вступ}
Спочатку опишемо, як зображення задається у цифровому вигляді (в пам'яті комп'ютера).
Зображення можна розуміти як матрицю. Якщо зображення чорно-біле, то кожна його точка задається 
1-байтним числом - від 0 до 255. 
Це число - значення інтенсивності точки, 0 відповідає рівню чорного (найтемніший), 255 - білого (найсвітліший). 
Для кольорового зображення потрібно зберігати 3 значення інтенсивності для каналів червоного, 
зеленого та синього, згідно із стандартом RGB.




\begin{definition}
 Нехай $x_{min}, x_{max}$ - максимальне та мінімальне значення інтенсивності відповідно.
 
 {\bfseries Діапазон інтенсивності зображення} - це проміжок $[x_{min}, x_{max}]$, 
 в межах якого лежать інтенсивності усіх точок (пікселів) зображення.
\end{definition}

Однак границі діапазону зображення можуть змінюватися лише в межах від 0 до 255. 
Будемо називати ці числа {\itshape граничними значеннями яскравості діапазону}.

\section{Лінійне контрастування зображення}

Задача контрастування вирішує проблему узгодженості діапазону 
зображення із оригінальним.

Приклад: нехай $x_{min}, x_{max}$ лежать далеко від границь діапазону. 
Тоді зображення буде виглядати ненасиченим, неконтрастним, ніби в тумані. 
Багато відтінків світла і темряви вже не розрізняються.

Нехай ми знаємо, до яких границь потрібно розширити діапазон - $[y_{min}, y_{max}]$.
При лінійному контрастуванні ми використовуємо лінійне перетворення, щоб спроектувати початковий діапазон на бажаний:

\begin{align}
y_{min} &= ax_{min} + b \\
y_{max} &= ax_{max} + b
\end{align}

Отриману систему з невідомими $a, b$ розв'язуємо і отримуємо формулу проектування:
\[
y = f(x) = \frac{y_{max} - y_{min}}{x_{max} - x_{min}}(x - x_{min})  + y_{min}  
\]
Тут $x$ - початкова інтенсивність, $y$ - кінцева.

Одним із рішень є максимально розширити діапазон зображення $[x_{min}, x_{max}]$. 
Це означає, що потрібно спроектувати його на діапазон зміни однобайтних чисел $[0, 255]$.



\begin{equation}
    y = \frac{255}{x_{max} - x_{min}}(x - x_{min})
\end{equation}


\begin{figure}[H]
    \centering
    \includegraphics[scale=0.7]{{img/02/01}.png}
    \caption{Приклад: до і після контрастування}
    \label{fig:3.1}
\end{figure}

\section{Соляризація зображення}

\begin{definition}
    Динамічний діапазон - це
\end{definition}

При даній обробці перетворення має вигляд:
\[
    y = k\cdot x \cdot (x_{max} - x)
    \]

де $x_{max}$ - максимальне значення початкового зображення,
$k$ - константа, яка задовольняє керування динамічним діапазоном.

Ця функція є квадратичною параболою. Її графік при $k = 1$ має вигляд:

\begin{figure}[H]
    \centering
    \includegraphics[scale=0.7]{{img/02/02}.png}
    \caption{Функція перетворення при $k = 1$}
    \label{fig:3.2}
\end{figure}

Коефіцієнт $k$ відповідає за керування діапазоном зміни інтенсивності після перетворення.

Отже, точки із прообразом $\frac{x_{max}}{2}$ після перетворення буде найосвітленішою.
А точки, які відповідали максимальній освітленості, після перетворення матимуть інтенсивність 0.

\begin{figure}[H]
    \centering
    \includegraphics[scale=0.7]{{img/02/03}.png}
    \caption{Приклад соляризації}
    \label{fig:3.3}
\end{figure}

Призначення соляризації: зменшити інтенсивність світлих відтінків та збільшити інтенсивність середніх відтінків.
Рівень білого після соляризації отримують області початкового зображення.

Після проведення соляризації потрібно зробити лінійне контрастування на $[0, 255]$. Якщо цього не зробити, то ми можемо вийти за межі 255, 
і адаптер спроектує їх на 255, і перетворення буде іншим.

\begin{figure}[H]
    \centering
    \includegraphics[scale=0.7]{{img/02/04}.png}
    \caption{Приклад соляризації}
    \label{fig:3.4}
\end{figure}

\section{Препарування зображення}
Препарування зображення - це цілий клас елементарних перетворень 
інтенсивності зображень.
Кожне перетворення має свої характеристики і сценарії застосування.

Наприклад, бінаризація зображення. Ми переходимо від grayscale
кольору до гами із двох кольорів - чорного і білого. Це перетворення застосовується, наприклад, при задачу потоншення контурів.

\begin{figure}[H]
    \centering
    \includegraphics[scale=0.7]{{img/02/05}.png}
    \caption{Функція перетворення бінаризації}
    \label{fig:3.5}
\end{figure}

Наступне препарування діє таким чином:
\begin{enumerate}
\item  Область із середньою зміною інтенсивності потраплять в область світлого.
\item Всі інші точки не зазнають перетворення
\end{enumerate}

Розглянемо ще одне препарування, функція якого має вигляд:
\begin{figure}[H]
    \centering
    \includegraphics[scale=0.7]{{img/02/07}.png}
    \label{fig:3.7}
\end{figure}
Воно застосовується, щоб побачити детальніше зміни інтенсивності в середніх відтінках.

Ще один приклад застосування бінаризації - отримання відбитку пальця.
\begin{figure}[H]
    \centering
    \includegraphics[scale=0.7]{{img/02/08}.png}
    \caption{Приклад бінаризації зображення}
    \label{fig:3.8}
\end{figure}

\section{Контрастність зображення}
Припустимо, ми вирішуємо задачу реставрації зображення.
Нехай функція реставрації залежить від деяких параметрів.
Нам потрібно підібрати такі значення цих параметрів, при яких зображення
буде найбільш контрастним.

\begin{definition}
{\bfseries Контрастність зображення} визначається за формулою:

\[
C(f) = \frac{1}{|\Omega|} \sum_{(x, y) \in \Omega} |f(x, y) - \mu_{A(x,y)}|     
\]

Тут $|\Omega| = N\cdot M$ - потужність носія $\Omega$, який
містить всі пікселі зображення.
$N, M$ - кількість стовпчиків і рядків зображення $f(x, y)$, $\mu_{A(x,y)}$ - середнє 
значення інтенсивності точок, сусідніх із точкою $(x, y)$. 

\end{definition}

\section{Медіанна фільтрація}
Що таке фільтрація? Фільтрація - це згладження різких перепадів яскравості зображення.
Лінійні фільтри добре подавляють шум, що близький до гаусівського.
Якщо є шум не є гаусівським, але має імпульсний характер, то такі фільтри вже неефективні.
Вдалим розв'язком такої задачі є застосування {\itshape медіанного фільтра}.

Медіанний фільтр являє собою евристичний метод обробки інформації. 
Алгоритм не є математичним розв'язком строго сформульованої задачі.

Імпульсний шум - на білому фоні є невеликі плямки чорного, або навпаки.

\subsection{Опис алгоритму медіанного фільтра}

Відбувається обробка кожної точки зображення. 

Потрібно обрати двовимірне вікно (апертура фільтра).
\begin{definition}
    {\bfseries Апертура фільтра} для заданої точки $(x,y)$ визначає 
    форму та розмір вікна для відбору точок-сусідів, які
    будуть задіяні в кроці алгоритму медіанного фільтра.
\end{definition}

Найчастіше обирається вікно у вигляді хреста або квадрата:

\begin{figure}[H]
    \centering
    \includegraphics[scale=0.7]{{img/02/09}.png}
    \caption{Можливі варіанти вибору вікна}
    \label{fig:3.9}
\end{figure}

Розміри апертури - це параметри, які потрібно підбирати для кожного зображення.
 
{\bfseries Алгоритм}
\begin{enumerate}
\item Фіксуємо точку зображення $(x, y)$
\item Нехай $\{y_1, y_2, \ldots, y_n \}$ - робоча вибірка, де $n$ - розмір вікна. Як правило, застосовують вікна з непарним числом точок $n$.
\item $\overline{y}$ - медіана вибірки, є продуктом фільтрації для точки $(x, y)$.
\end{enumerate}


Якщо імпульсний шум не є точковим, а покриває деяку область зображення, 
то він також може бути подавлений.
Однак потрібно вибрати апертуру вдвічі ширшу, ніж ширина імпульсного шуму (по вертикалі і горизонталі).

\begin{figure}[H]
    \centering
    \includegraphics[scale=0.7]{{img/02/10}.png}
    \caption{Подавлення імпульсного шуму за допомогою медіанного фільтру}
    \label{fig:3.10}
\end{figure}

\section{Логарифмічне перетворення зображень}
Розглянемо таке перетворення інтенсивності:

\[
y = clog(1 + x)    
\]

Перетворення розтягує діапазон темних відтінків та стискає діапазон яскравих (білих) відтінків.
Після цього перетворення нам знову потрібно буде лінійно спроектувати
на діапазон $[0, 255]$, щоби підвищити контрастність.

\section{Степеневе перетворення}
Функція перетворення інтенсивності кожної точки має вигляд:

\[
y = cx^{\gamma}, \\ c, \gamma > 0    
\]
Або 
\[
y = c(x + \varepsilon)^{\gamma}  
\]

\begin{example}
   При проектуванні зображення на монітор або при виведенні на принтер використовується
   перетворення $y = x^{0.4}$ - так звана {\itshape гамма-корекція}.

   Якщо цього не зробити, на екран монітора виведеться або дуже вибілене, або дуже затемнене зображення.
\end{example}

\section{Обробка зображень нелінійними перетвореннями}
Будемо розглядати унітарні перетворення - такі як перетворення Фур'є, Адамара.
Такі перетворення дозволяють представити функцію, яка описує зображення,
у вигляді сукупності спектральних коефіцієнтів, які відповідають окремим характеристикам зображення.

\begin{proposition}
   Перша спектральна складова пропорційна середній яскравості зображення.
   Складові більш високих просторових частот є мірою ``порізаності'' (швидкій зміні інтенсивності в близьких точках) даного зображення.

\end{proposition}
Ці властивості можна використати для покращення зображення.

Нехай $F(u, v)$ -дискретне перетворення Фур'є початкового зображення $f(i, j)$.

\begin{align}
    F(u, v) &= \sum_{j=0}^{N-1} \sum_{k=0}^{N-1} f(j, k) A(j, k, u, v) \\
    f(j, k) &= \sum_{u=0}^{N-1} \sum_{v=0}^{N-1} F(u, v) B(j, k, u, v)
\end{align}
де $A(j, k, u, v), B(j, k, u, v)$ - ядра прямого і оберненого перетворень.

\subsection{Корінь із спектральних коефіцієнтів}

Кожен із спектральних коефіцієнтів підноситься до степені, причому фаза коефіцієнтів зберігається.

\[
\overline{F}(u, v) = \frac{F(u, v)}{|F(u, v)|} |F(u, v)|^{\alpha} = F(u, v)|F(u, v)|^{\alpha-1}
\]

Тут $F(u, v) = |F(u, v)| e^{i\Phi(u, v)}$

Для спектра Фур'є маємо:
\[
    F(u, v) = M(u, v) e^{i\Phi(u, v)} 
\]
де $M(u, v)$ - амплітуда, $\Phi(u, v)$ - фаза.

Видозмінений коефіцієнт має вигляд:
\[
    \overline{F}(u, v) = \left(M(u, v)\right)^{\alpha} e^{i\Phi(u, v)} 
\]

Якщо $\alpha < 1$, то великі коефіцієнти зменшуються, 
а малі - збільшуються. Такий перерозподіл енергії в 
частотній площині призводить до більш ефективного 
використання діапазону зображення.

\subsection{Узагальнений кепстр}

Цей метод покращення зображення з нелінійними перетвореннями базується на обчисленні логарифма спектральних коефіцієнтів.

\[
\overline{f}(i, j) = M \sum_{j=0}^{N-1} \sum_{k=0}^{N-1} \left[ln(a + b|F(u, v)|) \right]  \frac{F(u, v)}{|F(u, v)|}   
\]

Ми розширимо діапазон невеликих коефіцієнтів Фур'є. Як правило, це високі частоти, які відповідають за швидку зміну зображення.

Таке перетворення застосовується для підкреслення контурів.

\begin{figure}[H]
    \centering
    \includegraphics[scale=0.7]{{img/02/11}.png}
    \caption{Кепстр зображення}
    \label{fig:3.11}
\end{figure}


Якщо записати дискретне перетворення Фур'є

\begin{align}
  f(k, s) &= \frac{1}{N} \sum_{u=0}^{N-1} \sum_{v=0}^{N-1} F_f(u, v) w^{-uk+sv}, \ \ w = e^{-t\frac{2\pi}{N}} \\
  F_f(u, v) &= \frac{1}{N} \sum_{k=0}^{N-1} \sum_{s=0}^{N-1} f(k, s) w^{uk+sv}
\end{align}

Тоді функція кепстра над зображенням має вигляд:

\begin{align}
\overline{f}(k, s) &= \frac{1}{N} \sum_{u=0}^{N-1} \sum_{v=0}^{N-1}  ln(a + b|F_f(u, v)|)  e^{i\Phi(u, v)} w^{-uk+sv} \\
\Phi(u, v) &= \frac{\mathrm{Im} F_f(u, v)}{\mathrm{Re} F_f(u, v}
\end{align}